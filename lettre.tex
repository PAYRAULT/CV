\documentclass[a4paper,12pt]{lettre}
 
\usepackage[activeacute,activegrave, french]{babel}
\usepackage[T1]{fontenc}
%\usepackage{lmodern}
 
\begin{document}
 
\begin{letter}{
    CNAM - DRH\\
    Case 4DGS03\\
    292 rue Saint Martin\\ 
    75141 PARIS CEDEX 03}
  
\date{le 20 avril 2015} 
 
\def\concname{Objet :~}
\conc{Candidature PRCM 0133}

\opening{Madame, Monsieur,}
 
 Je suis actuellement Responsable du D\'epartement "Ing\'enierie
 Safety" (Head of RAM-S Department) au sein de la division ferroviaire
 de Thales Communication et Syst\`eme (TCS). Mon d\'epartement est
 donc en charge d'effectuer l'ensemble des analyses de fiabilit\'e
 (RAM) et de s\'ecurit\'e (S) des syst\`emes et logiciels critiques
 embarqu\'es \`a bord des mat\'eriels ferroviaires r\'ealis\'es par
 Thal\`es. Le niveau \'elev\'e de certification requis pour leur mise
 en exploitation exige l'utilisation de m\'ethodes formelles pour
 construire leur cycle de vie complet. Je suis donc conduit \`a
 utiliser ou \`a superviser l'utilisation de m\'ethodes formelles
 telles que analyses statiques, interpr\'etations abstraites,
 v\'erifications par mod\`eles et r\'ealisation ou v\'erification de
 preuves (Coq, Isabelle, HOL, atelier B). 

De plus, les nouvelles obligations l\'egales du domaine ferroviaire
(Organisme d'Importance Vitale selon la PLM 2014-2019) m'impose de
certifier les architectures de cyber-s\'ecurit\'e (Security) pour nos
produits tout en garantissant le niveau de s\'ecurit\'e fonctionnelle
(Safety). Je suis \'egalement agr\'e\'e par l'Agence Nationale de la
S\'ecurit\'e des Syst\`emes d'Information (ANSSI) en tant qu'expert
m\'ethodes formelles (niveaux EAL5 \`a 7). Je r\'ealise donc des
expertises de syst\`emes d'exploitation sur cartes \`a puce en tant
que membre du CESTI de Thales, en utilisant majoritairement des
syst\`emes d'aide \`a la preuve.

J'ai la fiert\'e d'\^etre Ing\'enieur CNAM. J'ai en effet effectu\'e
mes \'etudes sup\'erieures (du cycle A au cycle C) au CNAM de Paris,
dans le domaine du logiciel en m'int\'eressant plus particuli\`erement
aux m\'ethodes formelles. J'ai obtenu le DEA Informatique Th\'eorique
Calcul Programmation (Paris VII) puis j'ai effectu\'e une th\`ese sous
la direction du Professeur Th\'er\`ese Hardin, soutenue en 2011, \`a
l'Universit\'e Pierre et Marie Curie. Ce troisi\`eme cycle s'est
effectu\'e tout en conservant une activit\'e
professionnelle. Parall\`element, j'ai aid\'e les Professeurs Hardin
et Vigui\'e-Donzeau-Gouge \`a concevoir le DESS puis le Parcours
``D\'eveloppement des Logiciels S\^urs'' co-habilit\'es par le CNAM-UPMC
et j'y ai \'et\'e responsable du cours de tests. Je garde des contacts
forts avec le monde universitaire par le biais de ma participation \`a
des cours magistraux, la co-r\'edaction d'articles de recherche et la
r\'ealisation de contrats.

Par la pr\'esente, je souhaite vous soumettre ma candidature aux poste
de Professeur Titulaire de Chaire de la chaire ``S\'ecurit\'e
Informatique''.  Le profil de ce poste semble correspondre \`a ma
double comp\'etence S\'ecurit\'e fonctionnelle et S\'ecurit\'e
informatique acquise au cours de ces vingt derni\`eres ann\'ees. Je
connais bien le fonctionnement de l'\'etablissement pour en avoir
\'et\'e \'el\`eve puis charg\'e de cours et de TD et je suis tout \`a
fait pr\^et \`a prendre en charge les t\^aches administratives li\'ees \`a
cette fonction.

%Je reste \`a votre disposition pour convenir d'un rendez-vous pour que
%je puisse vous pr\'esenter ma candidature plus en d\'etail et mieux
%comprendre le processus formel de d\'epôt de candidature. Je suis
%particuli\`erement disponible les mercredi et jeudi. 
 
\closing{Je vous remercie par avance de l'int\'er\^et que vous pourrez
  porter \`a ma candidature et je vous prie de recevoir, Madame,
  Monsieur, mes sinc\`eres salutations.}

\encl{\ \\
  - Pr\'esentation des titres et travaux\\
  - CV\\
  - Projet pour la chaire
}

\end{letter}
\end{document}
