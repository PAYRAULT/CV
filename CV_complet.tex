\documentclass[a4paper,12pt]{article}

\usepackage[activeacute,activegrave,french]{babel}
\usepackage[T1]{fontenc}
   
\usepackage{times}
\usepackage{amsmath}
\usepackage{amsfonts}
\usepackage{amssymb}

%\setlength{\topmargin}{0mm}
%\setlength{\headheight}{0mm}
%\setlength{\headsep}{0mm}
%\setlength{\textheight}{225mm}
%\setlength{\oddsidemargin}{0mm}
%\setlength{\textwidth}{160mm}
%\usepackage{supertabular}

% Added TRSS
%\usepackage{color}
%\usepackage{verbatim}
 
\pagestyle{plain}        

\newcommand{\Lustre}{{\tt Lustre}}
\newcommand{\Lezard}{{\tt Lezard}}
\newcommand{\Coq}{{\tt Coq}}
\newcommand{\Ocaml}{{\tt Ocaml}} 
\newcommand{\Modulog}{{\tt Modulog}}
\newcommand{\Prolog}{{\tt Prolog}}
\newcommand{\Focal}{{\tt Focalize}}

\newcommand{\smallspace}{\vspace{0.25cm}}
\newcommand{\bigspace}{\vspace{0.5cm}}

\title{Curriculum Vitae d\'etaill\'e}
\author{Philippe  Ayrault}
%\date{6/12/2010}                  

\begin{document}

%\def    \trss#1         {\textcolor{blue}{#1}}

%\setcounter{secnumdepth}{0}
%\setcounter{tocdepth}{0}

%-------------------%
% Debut du document %
%-------------------%
\maketitle


\section{\'Etat civil}
\begin{tabular}{l}
\underline{Nom de famille} : \textsc{AYRAULT} \\
\underline{Pr\'enoms} : \textsc{Philippe, Andr\'e, Eug\`ene, Simon} \\
\underline{Nationalit\'e} : \textsc{Fran\c{c}ais}\\
\underline{Date de naissance} : \textsc{le 20 juillet 1966 \`a
  Argenteuil (95)} \\
\underline{Situation de famille} : \textsc{Vie maritale, deux enfants} \\
\underline{Adresse} : \textsc{43, All\'ee du pont des Beaunes, 91120
  Palaiseau} \\
\underline{Mail} : philippe.ayrault@etersafe.com \\
\underline{T\'el\'ephone} : 01 60 10 71 43 \\
\underline{Mobile} : 06 86 03 06 25 \\

\end{tabular}


\section{R\'esum\'e}


\begin{itemize}
\item Expert Safety ferroviaire, spatial, nucl\'eaire et automobile
\item Membre du CESTI Thales (Security)
\item Plus de 20 ans d'exp\'erience
\item Gestion de projets nationaux et internationaux
\item Responsable du d\'epartement Safety \`a Thales Syst\`emes
  Ferroviares Int\'egr\'es
\item Recherche acad\'emique
\end{itemize}

\newpage
\section{Formation initiale}

\hspace{-0.6cm}\underline{DUT Mesures Physiques}, Universit\'e Saint-Denis, Juin 1986.

\smallspace
\hspace{-0.6cm}\underline{Dipl\^ome d'Ing\'enieur CNAM} en Informatique, Mai 1994.
\\
\underline{Sujet} : S\'emantique, Implantation et tests de Modulog.
\\
\underline{Jury} :
\begin{tabular}{lll}
  Mme V\'eronique Donzeau-Gouge Vigui\'e & CNAM & Pr\'esident  \\
  Mr Louis Dewez & CNAM & Rapporteur \\
  Mr Philippe Blot & & Examinateur \\
  Mr Jean-Fran\c{c}ois Dazy & CNAM & Examinateur \\
  Mr Jean-Pierre Tramecon & Thomson RGS & Examinateur industriel \\
\end{tabular}

\smallspace
\hspace{-0.6cm}\underline{Dipl\^ome d'Etudes Approfondies} Informatique Th\'eorique,
Calcul et Programmation, Universit\'e Paris 6 (LIP6), Responsable
J. Sakarovitch, Septembre 1994.

\smallspace
\hspace{-0.6cm}\underline{Doctorat en Informatique}, Universit\'e Paris 6 (LIP6),
Avril 2011. 
\\
\underline{Sujet} : D\'eveloppement de logiciel critique en \Focal\ -
M\'ethodologie et outils pour l'\'evaluation de conformit\'e.
\\
\underline{Jury} :
\begin{tabular}{lll}
  Mme Marie-Laure Potet & VERIMAG & Rapporteur \\
  Mr Claude Kichner & INRIA & Rapporteur \\
  Mme Th\'er\`ese Hardin & LIP6 & Directeur \\
  Mr Fran\c{c}ois Pessaux & & Directeur \\
  Mme V\'eronique Vigui\'e Donzeau-Gouge & CNAM & Examinateur \\
  Mr Christian Queinnec & LIP6 & Examinateur \\
  Mr Pierre-Etienne Moreau & LORIA & Examinateur \\
  Mr Richard Imhoff & Thales Rail Signaling & Invit\'e industriel\\
\end{tabular}


\newpage
\section{Activit\'es de recherche}
J'ai commenc\'e mes activit\'es de recherche lors de mon cycle C du
CNAM. J'ai pris une ann\'ee sabatique pour int\'egrer l'\'equipe de
Jean-Louis Dewez. J'ai travaill\'e sur le langage \Modulog, un Prolog
int\'egrant la notion de modules. Le langage a \'et\'e d\'evelopp\'e
par une succession d'\'etudiants sans r\'eel strat\'egie de tests pour
s'assurer de la non-regression des fonctionnalit\'es existantes au fur
et \`a mesure de l'ajout de nouvelles composantes du langage. J'ai
commenc\'e par d\'efinir la s\'emantique op\'erationnelle
d\'eclarative du langage \Modulog. Cette s\'emantique constitue la
base d'une sp\'ecification ex\'ecutable du langage et permet un
prototypage en Prolog de l'interpr\'eteur \Modulog.  Ensuite, j'ai
d\'efini un processus de tests syst\'ematiques et automatis\'e du
langage. J'ai d\'evelopp\'e un outillage int\'egrant le moteur du
langage afin de permettre de r\'ecup\'erer les sorties des cas de
tests et les comparer automatiquement \`a des r\'esultats
pr\'ed\'efinis. \`A partir de la sp\'ecification formalis\'ee du
langage, j'ai d\'evelopp\'e plus de 1000 cas de tests, permettant
d'identifier et de corriger environ 350 anomalies. Ce travail m'a
permis d'int\'egrer des m\'ethodes acquises lors de mes activit\'es
professionnelles sur un projet acad\'emique \cite{Ing-PAyrault}.
\\
Durant mes activit\'es S\^uret\'e de Fonctionnement \`a Surlog, j'ai
eu besoin de d\'evelopper un outil d'analyse de code sp\'ecifique \`a
mon m\'etier. Je me suis rapproch\'e de l'\'equipe SPI du LIP6
dirig\'ee par le Professeur Th\'er\`ese Hardin pour v\'erifier la
``soundness'' de l'outil d\'evelopp\'e.  Cette collaboration a
continu\'e sur plusieurs ann\'ees pour aboutir \`a l'encadrement d'une
th\`ese CIFRE. Le sujet concernait l'analyse des classes
d'\'equivalence entre composants logiciel. Une classe d'\'equivalence
repr\'esente l'ensemble des valeurs des entr\'ees d'un composant
g\'en\'erant le m\^eme comportement en sortie. L'ensemble des classes
d'\'equivalence d'un composant repr\'esente une partition sur le
produit cart\'esien des domaines des entr\'ees du composant. Il peut
\^etre \'egalement interpr\'et\'e comme la sp\'ecification en bo\^ite
noire du composant. Les classes d'\'equivalence \'etaient
repr\'esent\'ees par un syst\`eme de contraintes et la r\'esolution
entre composants \'etait r\'ealis\'ee par un syst\`eme
d'interpr\'etation abstraite sur les contraintes (artihm\'etique sur
les intervalles).  La th\`ese a \'et\'e soutenue par Fabrice Parrennes
en 2002 \cite{Phd-FParrennes}.
\\
En 2005, j'ai voulu compl\'eter mon cursus acad\'emique en
commencant une th\`ese sur 2 aspects du langage \Focal. Ce travail
\'etait encadr\'e par le Professeur Th\'er\`ese Hardin. Le premier
aspect concerne la d\'efinition d'un cycle de d\'eveloppement pour
l'outil \Focal\ et sa conformit\'e aux principales normes de
S\^uret\'e de Fonctionnement du logiciel \cite{TAP09, TTSS08}. Le
second aspect concerne la sp\'ecification formelle en \Coq\ du calcul
de d\'ependance entre Entr\'ees et Sortie d'un composant logiciel
\'ecrit en \Focal. La preuve de correction de la sp\'ecification a
\'t\'e r\'ealis\'ee en \Coq\ et un prototype a \'et\'e d\'evelopp\'e
en \Ocaml. Cette th\`ese a \'et\'e une grande exp\'erience
personnelle. Elle m'a permis de r\'efl\'echir sur mon m\'etier
d'\'evaluateur de syst\`emes critiques pilot\'es par du logiciel et de
tenter de r\'epondre aux besoins th\'eoriques et de formalisation des
analyses de risques \cite{Phd-PAyrault}.
\\
Les travaux de ma th\`ese ont \'et\'e poursuivis par la th\`ese de Vincent
Benayoun dirig\'ee par Catherine Dubois au CNAM et soutenue en 2012
\cite{Phd-VBenayoun, SEFM2012}.

\section{Activit\'es d'enseignement}
\hspace{-0.6cm}\underline{Parcours "D\'eveloppement de Logiciels
  S\^urs" - LIP6/CNAM}
\\
Responsable de la valeur ``Tests du logiciel'' de 2000 \`a
2007.  Les syst�mes critiques � logiciel pr�pond�rant doivent 
assurer la s�curit� des personnes ou de l'environnement m�me en cas 
de d�faillances. Ces d�faillances ont diff�rentes sources: erreur de 
sp�cification, erreur de conception, anomalies, d�faillance du
mat�riel supportant le logiciel, d�faillance des �quipements en
interface avec le logiciel... L'enseignement de la S�ret\'e de
Fonctionnement du logiciel n'existait pas au LIP6 malgr\'e la demande
du march\'e d'avoir des ing\'enieurs form\'es � cette discipline.
Dans le cadre de ce parcours, j'ai mont\'e la valeur de tests du
logiciel. Le test repr\'esente au alentour de 50\% du co�t de
d\'eveloppement d'un logiciel critique. Il est de plus soumis � une
d�monstration de couverture par l'ensemble des normes applicables dans
le domaine. Bas\'e sur mon exp\'erience de v\'erificateur et de
valideur, la valeur pr\'esentait les objectifs du tests et l'ensemble
du cycle de tests (tests unitaires, tests d'int\'gration et tests de
validation) ainsi que les diff\'erentes couvertures associ\'ees.
\'A la fin de la valeur, les \'etudients \'etait capable de relire
et/ou de r\'ediger un Plan de Tests pour un logiciel critique et le
mettre en place.

\smallspace
\hspace{-0.6cm}\underline{Approche s\'emantique des langages - CNAM}
\\
Charg\'e de Travaux Dirig\'es de 1993 \`a 1998. La valeur consistait �
fournir les outils pour la compr\'ehension d'un langage informatique.
\`A partir de la logique bool\'eenne, nous pr\'esentions la notion de
s\'emantique op\'rationnelle. Puis graduellement, nous augmentions la
complexit\'e des r\`egles pour aboutir \`a l'impl\'ementation en
\Prolog\ d'un interpr\'eteur \Prolog\ simplifi\'e. L'impl\'ementation
de cet interpr\'eteur \'etait utilis\'ee comme sujet de Travaux
Pratiques.  La fin de l'ann\'ee \'etait consacr\'ee \`a une
introduction \`a la logique de Hoare et \`a la preuve de programme
simple.
\\
Cette valeur \'etait dirig\'ee par Jean-Louis Dewez.


\newpage
\section{Parcours professionnel}
\subsection{Thales TCS - Syst\`emes Ferroviaires Int\{egr\'es, 2007 - }
\hspace{-0.6cm}\underline{Fonctions} : 
\begin{tabular}{l}
Head of Safety Department (depuis Juin 2012)\\
Resp. s\'ecurit\'e Projet Ligne 13 (2007-2012) \\
\end{tabular}

\hspace{-0.6cm}\underline{Effectif} : 30 personnes
\\
\subsubsection{Head of Department Engineering Safety}
Bla bla bla...

\subsubsection{Projet RATP Ligne 13}
En 2007, Thales TCS rach\`ete la branche ferroviaire d'Alcatel et le
projet RATP Ligne 13 red\'emarre sur une localisation fran\c{c}aise des
activit\'es. 

\subsubsection{CESTI}
En 2011, le CESTI de Thales est s\'electionn\'e par l'ANSSI pour
r\'ealiser l''\'evaluation au niveau EAL7 (le plus \'elev\'e des
Crit\`eres Communs) de la smart Card de Gemalto. Cette \'evaluation
consiste \`a montrer que la Machine Virtel (VM) en charge de
l'isolation entre les applications pr\'esentes sur la carte \'etait
conformes au Profile de Protection sp\'ecifications (U)SIM Java Card
Platform.  Les applications s\'ecuritaires pourront coexister, dans
les cartes certifi\'ees, avec des applications non s\'ecuritaires (ces
derni\`eres ne seront pas certifi\'ees, mais elles devront n\'eanmoins
satisfaire aux contraintes impos\'ees par la plateforme).  Pour
atteindre ce niveau d'\'evaluation, Trusted Labs, le partenaire de
Gemalto, a d\'evelopp\'e un mod\`ele formel complet de la machine
virtuel de sa sp\'ecification jusqu'\`a son code. Le langage utilis\'e
pour ce mod\`ele \'etait \Coq.  J'ai \'et\'e en charge de la
r\'ealisation de la partie formelle de l'\'evaluation. Il s'agissait
de v\'erifier la conformit\'e du mod\`ele \`a la sp\'ecification
JavaCard de SUN Technology ainsi que des propri\'et\'es de s\^uret\'e
d\'ecrites dans le Profil de Protection. Puis de v\'erifier le
mod\`ele de raffinement entre les diff\'erents niveaux de conception
mis en \oe{}uvre (sp\'ecification, conception, conception
d\'etaill\'ee). Enfin de montrer la conformit\'e du processus de
formalisation aux contraintes des Crit\`eres Communs et \`a la note
[ANSSI-CC-NOTE.12] (Note d'application - Mod\'elisation formelle des
politiques de s\'ecurit\'e d'une cible d'\'evaluation).
\\
En 2013, Gemalto et Trusted Labs annoncaient la premi\`ere
\'evaluation d'un de leur produit au niveau EAL7 par l'ANSSI. Une \'etude
similaire a \'et\'e ensuite r\'ealis\'e avec les m\^emes partenaires
sur une carte \`a puce bas\'ee sur le noyau Multos. 

\subsection{Etersafe SARL,  2004 - 2006}
\hspace{-0.6cm}\underline{Fonctions} : Expert Technique ind\'ependant
\\
\subsubsection{Projet Phil\'eas}
En 2003, le Syndicat Mixte des Transports du Douaisis (SMTD) a opt\'e
pour une technologie encore inconnue en France pour la r\'ealisation
de son nouveau tramway : un v\'ehicule sur pneus \`a guidage
magn\'etique par le sol.  La mise en \oe{}uvre de ce type de tramway
semblait plut\^ot ais\'ee par rapport aux syst\`emes classiques de
tramway avec rails et cat\'enaires.  Des aimants plac\'es sous la
voie, forment un pattern des signaux qui sont lus par le v\'ehicule,
au moyen d'un syst\`eme automatique embarqu\'e.  Le guidage
magn\'etique et tous les essieux ind\'ependants du v\'ehicule en font
un v\'ehicule monotrace identique au rail.  La soci\'et\'e
n\'eerlandaise APTS (Advanced Public Transport Systems) m'a contact\'e
pour auditer la solution technique et mettre en place le processus
d'homologation du v\'ehicule. C'\'etait la premi\`ere fois que cette
soci\'et\'e d\'eployait un tel syst\`eme en milieu urbain.

Les solutions techniques ...

\subsubsection{Projet RATP Ligne 13}
En 2002, Alcatel Transport Toronto gagne le march\'e de r\'enovation
de la Ligne 13 du m\'etro parisien. Il s'agit de d\'eployer une
solution \`a base d'un CBTC (Communication Based Train Control).


\subsection{Surlog SA, 1994 - 2003}
\hspace{-0.6cm}\underline{Fonctions} : Directeur Technique (5 derni\`eres ann\'ees).
\\
\underline{Effectif} : 20 personnes
\\
\subsubsection{DomaineFerroviaire}
\begin{itemize}
\item Alstom: preuve formelle PAI, preuve formelle B (v\'erification
  de mod\`ele), Analyse de s\'ecurit\'e de syst\`eme CBTC-SOL, de la
  plateforme de s\'ecurit\'e 2oo3, .... 
\item RATP: D\'efinition de processus de v\'erification de logiciel
  d\'evelopp\'e suivant la m\'ethode formelle B, Certification du
  premier syst\`eme d'enclenchement informatis\'e  install\'e \`a  la RATP (PMI Porte des Lilas),
  Mini-m\'etro de l'a\'eroport de Paris CdG 
\item Faiveley: v\'erification de Portes Pali\`eres.  
\end{itemize}


\subsubsection {Auditeur}
D\'efinition, mise en place de processus de SdF et suivi de
l'application pour le compte de plusieurs clients... 

\subsubsection{S\'ecurit\'e Informatique}
\textbf{Hyperviseur Polyx\`ene}
Le projet SINAPSE (Solution Informatique \`a Noyau Avanc\'e Pour une S\^uret\'e Elev\'ee) 
lanc\'e en 2004 par la D\'el\'egation G\'en\'erale pour l'Armement (DGA) et 
dont le ma\^itre d'\oe{}uvre est la soci\'et\'e Bertin Technologies a permis de d\'evelopper l'hyperviseur Polyx\`ene.

Un hyperviseur est un logiciel qui permet \`a plusieurs syst\`emes d'exploitation de travailler en
m\^eme temps sur un seul ordinateur en assurant une \'etanch\'eit\'e compl\`ete entre ces syst\`emes. 
Polyx\`ene propose une architecture de tr\`es haute s\'ecurit\'e
permettant d'acc\'eder sur un ordinateur unique \`a des informations venant de r\'eseaux de niveau de
classification diff\'erent. En effet, la s\'eparation physique des r\'eseaux - garante des niveaux de
s\'ecurit\'e informatique - est une exigence devenue trop co\^uteuse ou trop contraignante, y compris
pour certains syst\`emes gouvernementaux. Polyx\`ene est une solution et peut \^etre mis en \oe{}uvre sur des postes
dit \emph{multi-sensibles}, sur des syst\`emes embarqu\'es et des syst\`emes nomades durcis. 
Dans le cadre de ce projet, je tenais le r\^ole de CESTI interne. J'ai d\'efini l'ensemble du processus de d\'eveloppement (r\`egles d'architecture, r\`egles de codage, processus de tests, gestion documentaire, gestion de configuration et des modifications, ...) 
des logiciels en conformit\'e avec les recommandations de la norme ISO/IEC 15408 (Crit\`eres Commun). J'ai assur\'e le suivi
de la mise en \oe{}uvre du processus et r\'ealis\'e l'ensemble des analyses de risques.  
Le challenge provenait que le projet SINAPSE d\'emarrait \`a partir d'un micro-noyau existant qui n'avait jamais fait l'objet d'une
certification et que les \'equipes de d\'eveloppement n'avaient jamais d\'evelopp\'e de logiciel \`a haut niveau de s\'ecurit\'e. Le processus d\'efini devait \'egalement minimiser les reprises sur le produit existant tout en garantissant l'obtention du certificat 
en fin de projet.

En septembre 2009, Polyx\`ene est le premier hyperviseur \`a haut niveau de s\'ecurit\'e
fran\c{c}ais \`a obtenir un certificat de niveau de s\'ecurit\'e EAL 5 aupr\`es de l'Agence Nationale de la
S\'ecurit\'e des Syst\`emes d'Information (ANSSI).

\subsubsection{Spatial}
\textbf{Plateforme HYDRA.} La plateforme HYDRA a \'et\'e construite
\`a l'ESA (Agence Spatiale Europ\'eenne) \`a  Norgvek au Pays-Bas afin
de r\'ealier les essais de vibrations du satellite Envisat. La
plateforme  Il n'existait pas de table suffisamment puissance pour
faire vibrer un satellite de ce poids et de cette dimension. La
vibration consiste \`a appliquer sur un satellite l'ensemble des
contraintes physiques subies lors de son d\'ecollage
(acc\'el\'eration, rotation, vibrations). C'est le dernier test
environnemental r\'ealis\'e sur le satellite avant son transport sur
le site de lancement de Kourou. Plusieurs mois avant le lancement du
satellite, la NASA avait bris\'e un satellite lors des essais en
vibration, en le faisant vibrer à 20g au lieu de 2g. Ceci provoquera
la destruction compl\`ete du satellite (cisaillement), des dommages
financiers cons\'equent. L'ESA nous a donc
demand\'e de v\'erifier de nveau de s\'ecurit\'e atteint par le
logiciel de pilotage de la plateforme de vibration HYDRA. 

La plateforme HYDRA est compos\'ee de 8 v\'erins lui permettant de
d\'eplacer une charge utile jusqu'\'a 23 tonnes selon les 6 axes de
libert\'e (3 translations et 3 rotations). Des algorithmes complexes
sont mis en oeuvre pour transformer les demandes selon les 6 axes de
libert\'e  vers les 8 v\'erins. L'architecture est bas\'ee sur
un ensemble de processeurs travaillant en parall\`ele. Le travail
r\'ealis\'e a consist\'e \'a identifier, \'a partir du code source,
les diff\'erents fl\^ots de donn\'ees et de les comparer aux
protections existantes (logiciel ou mat\'eriel). Plusieurs anomalies
ont \'et\'e d\'ecouvertes et corrig\'ees avant la mise en service de
la plateforme. 

Le satellite Envisat a pass\'e les tests de vibrations avec succ\`es et
a \'et\'e mis en orbite par Ariane 5 en 2001. Un article a \'et\'e
publi\'e en  partenariat avec l'ESA lors du colloque DASIA 2001
\cite{DASIA2001}.


\smallspace
\textbf{ASEPC.} L'Automate de S\'ecurit\'e de l\'Etage Principal
Cryotechnique (ASEPC) de la fus\'ee Ariane 5 est l'automate
surveillant le remplissage en Hydrog\`ene Liquide (26 tonnes de LH2)
et en Oxyg\`ene Liquide (132 tonnes de LOX) des r\'eservoirs
principaux de la fus\'ee. Ces r\'eservoirs permettent d'alimenter le
moteur Vulcain du lanceur.  Lors de l'assemblage du lanceur sur la
plateforme de lancement, il est n\'ecessaire de pr\'e-remplir le
r\'eservoir principal afin de renforcer la structure du lanceur (le
poids du lanceur \'etant primordial, la rigidit\'e de la structure est
assur\'ee par la pression dans les r\'eservoirs). L'assemblage final
du lanceur et de sa charge utile s'effectue en hanger et dure environ
2 semaines.  Tout m\'elange des composants des 2 r\'eservoirs avant la
mise à feu peut mettre en danger les personnes installant la charge
utile et provoquer un s\'erieux endommagement de la plateforme de
tir. L'Agence Spacial nous a demand\'e de r\'ealiser un mod\`ele
formel de l'automate et de prouver des propri\'e\'es de s\'ecurit\'e.
La mod\'elisation a \'et\'e effectu\'e avec le langage \Lustre\ et la
preuve des propri\'et\'es de s\'ecurit\'e avec l'outil \Lezard.

\subsubsection{Automobile}
PSA, 

\subsection{Thomson RGS}
D\'eveloppement d'un terminal de messagerie s\'ecuris\'e
inter-arm\'ees. D\'eveloppement et validation sur plateforme de la
librairie des fonctions de s\'ecurit\'e informatique du terminal :
\begin{itemize}
  \item driver Unix pour acc\'eder \`a la carte de chiffrement.
  \item chiffrement/d\'echiffrement des messages
  \item protocole d'authentification du terminal
  \item effacement s\'ecuris`e des fichiers
\end{itemize}

\newpage
\bibliographystyle{plain}
\bibliography{bibli}

\end{document}
