
%%%% debut macro %%%%
\begin{filecontents}{block.sty}
\ifx\endBlock\undefined
\def\block{\begingroup%
\def\endblock{\egroup\endgroup}%
\vbox\bgroup}%
\long\def\Block{\begingroup%
\def\endBlock{\unskip\egroup\endgroup}%
\pagebreak[2]\vspace*{\parskip}\vbox\bgroup%
\par\noindent\ignorespaces}
\long\def\IBlock{\begingroup%
\def\endIBlock{\unskip\egroup\endgroup}%
\pagebreak[2]\vspace*{\parskip}\vbox\bgroup\par\ignorespaces}
\def\need#1{\ifhmode\unskip\par\fi \penalty-100 \begingroup
% preserve \dimen@, \dimen@i
   \ifdim\prevdepth>\maxdepth \dimen@i\maxdepth
      \else \dimen@i\prevdepth\fi
   \kern-\dimen@i
   \dimen@\pagegoal \advance\dimen@-\pagetotal % space left
   \ifdim #1>\dimen@
        \vfill\eject\typeout{WARNING- EJECT BY NEED}
   \fi
   \kern\dimen@i
   \endgroup}
\def\lneed#1{\need{#1\baselineskip}}
% \begin{block} ... \end{block} delimite un bloc qui restera,
%                               si possible, sur une seule page.
\long\def\TBlock{\begingroup%
\def\endTBlock{\unskip\egroup\endgroup}%
\pagebreak[2]\vspace*{\parskip}\vtop\bgroup%
\par\noindent\ignorespaces}
\else
\typeout{block.sty already loaded}
\fi
\endinput

\def\need#1{\par \penalty-100 \begingroup
% preserve \dimen@, \dimen@i
   \ifdim\prevdepth>\maxdepth \dimen@i\maxdepth
      \else \dimen@i\prevdepth\fi
   \kern-\dimen@i
   \dimen@\pagegoal \advance\dimen@-\pagetotal % space left
   \ifdim #1>\dimen@ \vfil \eject \fi
   \kern\dimen@i
   \endgroup}
\end{filecontents}
