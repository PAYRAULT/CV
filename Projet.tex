\documentclass[a4paper,12pt]{article}

\usepackage[activeacute,activegrave,french]{babel}
\usepackage[T1]{fontenc}
\usepackage{times}
\usepackage{amsmath}
\usepackage{amsfonts}
\usepackage{amssymb}

%\setlength{\topmargin}{0mm}
%\setlength{\headheight}{0mm}
%\setlength{\headsep}{0mm}
%\setlength{\textheight}{225mm}
%\setlength{\oddsidemargin}{0mm}
%\setlength{\textwidth}{160mm}
%\usepackage{supertabular}

% Added TRSS
\usepackage{color}
\usepackage{verbatim}

\pagestyle{plain}        


\newcommand{\smallspace}{\vspace{0.25cm}}
\newcommand{\bigspace}{\vspace{0.5cm}}


\title{Projet}
\author{Philippe  Ayrault}
%\date{6/12/2010}                  

\begin{document}

\setcounter{secnumdepth}{0}
\setcounter{tocdepth}{0}

%-------------------%
% Debut du document %
%-------------------%
\maketitle

Mon projet ...

Les diff\'erents th\`emes :

\hspace{-0.6cm}\textbf{Management de la s\'ecurit\'e}
\\
Les diff\'erents aspects du m\'etier de Responsable SSI dans une entreprise.
\begin{itemize} 
  \item Connaissance du processus de s\'ecurit\'e
  \item Connaissance des diff\'erents r\^oles
  \item Assurer la protection des informations sensibles
  \item Assurer la conformit\'e avec les r\`egles de obligations
    gouvernementales, les normes  et le respect de la vie priv\'ee
  \item Connaissance des principes de bases de la s\'ecuri\'e
      \begin{itemize}
    \item Confidentialit\'e : Protection contre les divulgations
    \item Integrit\'e : Protection contre les alt\'erations
    \item Disponibilit\'e : Protection contre la destruction
    \item Authentification : Qui demande une requ\`ete ?
    \item Authorization : Quels sont les droits et les privil\`eges
      d'un utilisateur
    \item Auditabilit� : Capacit\'e � reconstruire l'historique des
      actions
    \item Management de la configuration, des sessions et des exceptions
    \end{itemize}
\end{itemize}

\smallspace
\hspace{-0.6cm}\textbf{Certification d'application et de produits}
\\
Les diff\'erents aspects du m\'etier de certificateur
d'applications. Peut-\^etre trop une niche ??
\begin{itemize}
  \item Bla bla bla...
\end{itemize}
 

\smallspace
\hspace{-0.6cm}\textbf{S\'ecurit\'e du logiciel}
\\
Les diff\'erents aspects du m\'etier de d\'eveloppeur de logiciel s�r dans une entreprise.
\begin{itemize}
  \item Attaques bas\'ees sur la m\'emoire (d\'ebordement pile,
    d\'ebordement m\'emoire, Return-Oriented Programming, ...) et leurs
    protections 
  \item S\'ecurit\'e des applications web (injection SQL, Cross-site
      scripting, Cross-site forgery, ...)
  \item Conception logiciel s�r, mod\'elisation formelle,  revue de
    code et tests
  \item s\'ecurit\'e des r\'eseaux
  \item Tests de p\'en\'etration
\end{itemize}

\smallspace
\hspace{-0.6cm}\textbf{Usable security}
\\
Les diff\'erents aspects du m\'etier de sp\'ecificateur d'application critique.
\begin{itemize}
  \item  Mental and psychological models, Usability
    studies, authentification
  \item Know your business and support it with secure solutions. Strong
    background in security technology but also a knowledge of the
    business of the customer 
\end{itemize}

\smallspace
\hspace{-0.6cm}\textbf{Cryptographie et contr\^ole d'acc\`es}
\\
Module sur les aspects th\'eoriques de la SSI.
\begin{itemize}
  \item Chiffrement par flot 
  \item Chiffrement par bloc
  \item G\'en\'eration d'al\'ea
  \item Chiffrement asym\'etrique
  \item Authentfication et non-r\'epudiation
  \item Politiques de s\'ecurit\'e
\end{itemize}

\smallspace
\hspace{-0.6cm}\textbf{La s\'ecurit\'e mat\'erielle}
\\
Les diff\'erents aspects du m\'etier de d\'eveloppeur de mat\'eriel s�r dans une entreprise.
\begin{itemize}
\item Vulnerabilit\'e du mat\'eriel
\item Attaques par canaux cach\'es
\item Attaques physiques
\item Trust Platform Modules (TPM)
\end{itemize}


\end{document}
